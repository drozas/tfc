\section{Coordination}

\begin{frame}
	\frametitle{Coordination} 
	The main mechanisms to coordinate with the rest of the teams were:
	\begin{itemize}
	  \item F2F: F2F meetings with members of  NTNU and Telenor teams during my
	  stay in Trondheim (Norway) and with my supervisor in Madrid. It was the
	  richest communication mechanism, and it was specially interesting for the
	  requirements elicitation processes.
	  \item Teleconferences: Weekly teleconference meetings with NTNU,
	  Telenor and CTI members. Very useful mechanism to coordinate 
	  the work in team, to discuss the state of the project, to share ideas, etc.
	  \item Wiki: Wiki website (http://www.astra-project.net/wiki)
	  to coordinate the bundles development process and to create documentation.
	\end{itemize}

\end{frame}


\begin{frame}

	\begin{itemize}

	  \item Subversion: The project code is hosted in a SVN server at NTNU
	  (http://basar.idi.ntnu.no/svn/astra/). Very useful to
	  coordinate the development process, since it allows us to avoid and resolve 
	  possible code conflicts, and to have revisions for every code updating.
	  \item E-mail: Useful to coordinate with all the members of the team
	  in an asynchronous way. For instance, it has been used to report bugs.
	\end{itemize}

\end{frame}
