%%%%%%%%%%%%%%%%%%%%%%%%% Coordination %%%%%%%%%%%%%%%%%%%%%%%

\section{Coordination}
\label{sec:coordination}
In this section we will discuss shortly the mechanisms we used in order to
coordinate with the rest of teams during the realization of this project.
This is specially interesting taking into account the big amount of partners
from several countries which are part of the ASTRA project (see Section
\ref{section:what-is-astra}).

The employed coordination mechanisms are listed and described briefly below:
\begin{itemize}
  \item F2F (Face To Face): I have assisted to F2F meetings with members of
  NTNU and Telenor teams during my stay in Trondheim (Norway) (see Section
  \ref{sec:contribution} to see time references). This is obviously the richest
  communication mechanism, and it was specially interesting for the requirements
  elicitation processes. This has also been the main coordination mechanism with
  my supervisor at URJC during my stay in Madrid (Spain).
  \item Teleconferences: I have had weekly teleconference meetings with NTNU,
  Telenor and CTI members. This has been a very useful mechanism to coordinate 
  the work in team, to discuss the state of the project, to share ideas, etc.
  \item Wiki: We have used a wiki website (http://www.astra-project.net/wiki)
  to coordinate the bundles development process. This has been a very useful
  tool to be aware about other teams bundles and to produce a good
  documentation.
  \item Subversion: The project code is hosted in a SVN server at NTNU \\ 
  (http://basar.idi.ntnu.no/svn/astra/). This has been a really useful tool to
  coordinate the development process, since it allows us to avoid and resolve 
  possible code conflicts, and to have revisions for every code updating.
  \item E-mail: This supposes the main asynchronous communication mechanism,
  and it has been very useful to coordinate with all the members of the team.
  For instance, it has been used to report bugs.
\end{itemize}

