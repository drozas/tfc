%%%%%%%%%%%%%%% Requirements %%%%%%%%%%%%%%%%%%%%%%%%%
\section{Requirements}
\label{sec:requirements}

In this section we will describe shortly the requirements that we gathered
during the several requirements elicitation processes performed during all the
iterations (see Table \ref{table:iterations}). 
It is interesting to remark that this process becomes even more important (and
challenging) in projects of a researching nature like ASTRA, due to the
continuous rise of requirements implicit in its kind.

Tables \ref{table:functional-requirements} and
\ref{table:non-functional-requirements} summarize the most important functional
and non functional requirements that we gathered respectively.
%%Functional requirements
\begin{table}[h!]
	\small
    \begin{center}
		\begin{tabular}{||r|l||}
		\hline \hline
		\multicolumn{2}{||c||}{\bfseries{Functional requirements}} \\
		\hline \hline
			1. & Create a repository to store awareness applications. \\
			\hline
			2. & The information to share can be customized by the user before being
			stored.\\
			\hline
			3. & The repository has to take into account the visibility of the
			applications\\
			   & in terms of communities.\\
			\hline
			4. & It is necessary to offer functionalities to browse the repository
			by communities.\\
			\hline
		 	5. & It is necessary to create a mechanism to search applications by
		 	different \\
		 	  & criteria (tags, description, type or any).	\\
			\hline
		 	6. & It is necessary to create a mechanism to recommend applications 
		 	based \\
		 	 & on the similarity with respect to another application.\\
			\hline
		 	7. & During the appropriation of an application, it is necessary to
		 	offer \\
		 	& the user the possibility of customizing the application (i.e.:
		 	choosing the rules).\\
		 	\hline
		 	8. & Create a mechanism to obtain a ``human readable'' description of a
		 	rule.\\
		 	\hline
		 	9. & Create a system which allows us to tag applications choosing the
		 	visibility:\\
		 	& public, communities or private.\\
			\hline
		 	10. & Create a GUI which allows the interaction with the repository 
		 	and perform \\
		 	& common operations (logging in, showing help, etc.).\\
		 	
		
		\hline \hline
		\end{tabular}
		\caption{\label{table:functional-requirements}Functional requirements}
	\end{center}
\end{table}


%%Non-Functional requirements
\begin{table}[h!]
	\small
    \begin{center}
		\begin{tabular}{||r|l||}
		\hline \hline
		\multicolumn{2}{||c||}{\bfseries{Non functional requirements}} \\
		\hline \hline
			1. & The functionalities have to be offered by OSGi bundles through
			their  \\
			& web services interfaces. \\
			\hline
			2. & The system has to perform the retrieving process as transparent
			 as possible \\
			& for the user. \\
			\hline
			3. & All the components have to be multiplatform. \\
			\hline
			4. & The GUI has to be intuitive.\\
			\hline
			5. & The GUI has to be easy to connect to other bundles in the future.\\
			\hline
		 	6. & It is necessary to study the possibility of using ontology services
		 	provided\\
		 	& by \verb|OntologyManager| in order to improve the performance of some\\
		 	& of the functionalities (i.e.: application adaptation).\\
		
		\hline \hline
		\end{tabular}
		\caption{\label{table:non-functional-requirements}Non functional requirements}
	\end{center}
\end{table}