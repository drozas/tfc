\chapter{Conclusions}
\label{chap:conclusions}
This chapter concludes the report discussing briefly the fulfillment of the
objectives and discussing my personal contribution to ASTRA. Some
suggestions about future work are also included, as well as a personal evaluation.


\section{Achieved goals}
\label{sec:goals}

In this report we have detailed the process carried on to develop and integrate
a set of bundles to manage awareness applications into ASTRA, including
functionalities for sharing, tagging, locating, appropriating and adapting the
applications. A new GUI to access this functionalities has also been
developed, demonstrating the flexibility of the ASTRA SOA (see Section 
\ref{subsubsec:tech-astra-soa}). All the work has been carried on stressing its
extensibility, so new functionalities can be easily added in the future as we
will see in Section \ref{sec:future-work}. Therefore, we can conclude we have 
successfully fulfilled all the goals stated in Chapter \ref{chap:objectives}.

\section{Contribution}
\label{sec:contribution}
In this section we will discuss briefly my contribution to ASTRA, including
also time references.

I started my contribution to ASTRA as part of my summer job for the IDI
(Institutt for Datateknikk og Informasjonsvitenskap) at NTNU during
the summer of 2008 (July-September). In this period I participated in the
analysis, design, implementation and testing of the first version of 
\verb|RepositoryManager|, \verb|TagManagerBackend| and \verb|TagManagerNode|.
During this term, I was working in close cooperation with another member of the
NTNU ASTRA team, and we followed a extreme-programming model (see Section 
\ref{section:methodology-spiral}). The work performed during this period made 
up iterations 1 and 2 (see Table \ref{table:iterations}).

I resumed my collaboration with ASTRA project in February 2009. During this
period I worked in the analysis, design, implementation and testing of\\
\verb|ApplicationManager| and in the extension of the functionalities of
other bundles: including search capabilities in \verb|RepositoryManager| and
the addition of new functionalities in \verb|TagManagerBackend| and
\verb|TagManagerNode|, etc. This made up iterations 3 and 4 (see Table
\ref{table:iterations}). Most of the work was developed during my stay in
Madrid, and I coordinated using the mechanisms explained in Section
\ref{sec:coordination}. I also had the opportunity to come back to Norway to
work in ASTRA during one month and a half during the summer, which was
extremely useful to elicit and implement the last requirements and conclude 
successfully the project.


\section{Future work}
\label{sec:future-work}
In this section we present a set of possible enhancements for this project:

\begin{itemize}
  \item The parameters for the searching by similarity process can be adjusted
  to increase its recall and precision once an evaluation process similar to
  the one explained in Section \ref{subsec:testing-search-engine} with real
  user data is carried on.
  \item The use of ontologies for helping in the application adaptation process
  can be included once the implementation of the necessary methods in the\\
  \verb|OntologyManager| is finished. \verb|ApplicationManager| is already
  prepared to make use of its services.
  \item An extension of the GUI could be performed, adding the possibility of
  publishing applications, join communities, rules edition, etc.
  \item New browsing methods (by owner, by tags, etc.) could be developed, to
  allow the user localize applications in new ways.
\end{itemize} 


\section{Personal evaluation}
My work for ASTRA has been one of the most enriching experiences of my
career. I have had the opportunity to work in a real researching project and
to collaborate with teams from several countries. It has also been really
enriching in technological terms, since I have increased my knowledge about all
the technologies discussed in Chapter \ref{chap:methodology}.
It has been really important in personal terms as well, since this experience
allowed me to discover my passion for researching.
Therefore, I will always be grateful for having had the opportunity to 
be part of this project.
