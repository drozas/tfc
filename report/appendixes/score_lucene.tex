\chapter{Document scoring in Lucene}
\label{appendix:score-lucene}
\phantomsection 

In this section we will explain the scoring process employed in our search
engine. We use the class \verb|org.apache.lucene.search.DefaultSimilarity|
provided by the Lucene library. 
This explanation is based on \cite{lucene-scoring-explanation} and
\cite{lucene-scoring-formula}.

Lucene scoring uses a combination of the Vector Space Model (VSM) of
Information Retrieval and the Boolean model to determine how relevant a given 
Document is to a User's query. In general, the idea behind the VSM is the more 
times a query term appears in a document relative to the number of times the
term  appears in all the documents in the collection, the more relevant that 
document is to the query. It uses the Boolean model to first narrow down the 
documents that need to be scored based on the use of boolean logic in the Query
specification. Lucene also adds some capabilities and refinements onto this
model to support boolean and fuzzy searching, but it essentially remains a VSM 
based system at the heart.

The score for a query \emph{q} in a document \emph{d} is computed as follows:

\[
score(q,d) =  coord(q,d) * queryNorm(q)  * \sum_{t\phantom{i}in\phantom{i}
q}(tf(t\phantom{i}in\phantom{i}d) * idf(t)^2 * t.getBoost() * norm(t,d))
\]

where
\begin{itemize}
  \item \emph{tf(t in d)} correlates to the term's frequency, defined as the
  number of times term \emph{t} appears in the currently scored document
  \emph{d}. Documents that have more occurrences of a given term receive a
  higher score. The default computation for \emph{tf(t in d)} in
  \verb|DefaultSimilarity| is: \[ tf(t\phantom{i}in\phantom{i}d) =
  frequency^{1/2}\]
  \item \emph{idf(t)} stands for Inverse Document Frequency. This value
  correlates to the inverse of \emph{docFreq} (the number of documents in which
  the term \emph{t} appears). This means rarer terms give higher contribution to
  the total score. The default computation for \emph{idf(t}) in
  \verb|DefaultSimilarity| is: \[ idf(t)= 1 + \log (numdocs/(docFreq + 1)) \]
  \item \emph{coord(q,d)} is a score factor based on how many of the query  
  terms are found in the specified document. Typically, a document that
  contains more of the query's terms will receive a higher score than another 
  document with fewer query terms. This is a search time factor computed
  at search time.
  \item \emph{queryNorm(q)} is a normalizing factor used to make scores between
  queries comparable. This factor does not affect document ranking (since all 
  ranked documents are multiplied by the same factor), but rather just attempts
  to make scores from different queries (or even different indexes) comparable.
  This is a search time factor computed at search time. The default computation
  in \verb|DefaultSimilarity| is: \[ queryNorm(q)= 1 /sumOfSquareWeights^{1/2}\]
  \\ The sum of squared weights (of the query terms) is computed by the query
  \verb|Weight object|. For example, a boolean query  computes this value as: \[
  sumOfSquareWeights= q.getBoost()^2 *  \sum_{t\phantom{i}in\phantom{i}
q}((idf(t) * t.getBoost())^2)\]
  \item \emph{t.getBoost()} is a search time boost of term \emph{t} in the query
   \emph{q} as specified in the query text.
  \item \emph{norm(t,d)} encapsulates a few (indexing time) boost and length
  factors:
  		\begin{itemize}
            \item \emph{Document boost}, set by calling \verb|doc.setBoost()| 
            before adding the document to the index. 
            \item \emph{Field boost}, set by calling \verb|field.setBoost()|
            before adding the field to a document. 
            \item \emph{lengthNorm(field)}, computed when the document is added
            to the index in accordance with the number of tokens of this field
            in the document, so that shorter fields contribute more to the score. 
            \emph{LengthNorm} is computed by the \verb|Similarity| class in
            effect at indexing.
          \end{itemize}
   When a document is added to the index, all the above factors are
   multiplied. If the document has multiple fields with the same name, all
   their boosts are multiplied together: \[
  norm(t,d)= doc.getBoost() * 
  lengthNorm(field) * \prod_{field\phantom{i}in\phantom{i}
  d\phantom{i}named\phantom{i}as\phantom{i}t}(f.getBoost()) \]
\end{itemize}



